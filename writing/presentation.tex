% Created 2015-05-26 Tue 14:36
\documentclass{article}
\usepackage[utf8]{inputenc}
\usepackage[T1]{fontenc}
\usepackage{fixltx2e}
\usepackage{graphicx}
\usepackage{longtable}
\usepackage{float}
\usepackage{wrapfig}
\usepackage{rotating}
\usepackage[normalem]{ulem}
\usepackage{amsmath}
\usepackage{textcomp}
\usepackage{marvosym}
\usepackage{wasysym}
\usepackage{amssymb}
\usepackage{amsthm}
\usepackage{hyperref}
\usepackage[citestyle=authoryear-icomp,bibstyle=authoryear,hyperref=true,backref=true,maxcitenames=3,url=true,backend=biber,natbib=true]{biblatex}
\tolerance=1000
\usepackage{kk-math}
\author{Egg Beaters}
\date{\textit{<2015-05-26 Tue>}}
\title{Theoretical Results}
\hypersetup{
  pdfkeywords={},
  pdfsubject={},
  pdfcreator={Emacs 24.4.1 (Org mode 8.2.10)}}
\begin{document}

\maketitle
\pagebreak

\section{Intermediate points of fixed points compatible words lie in A}
\label{sec-1}
Daniel

\pagebreak

\section{Coordinates of intermediate points}
\label{sec-2}
\begin{claim}
Given a fixed word \( \omega \in \mathbb{F}_2 \left\langle H,V \right\rangle  \) and a compatible homotopy class $\alpha$, the coordinates of fixed points of \( \Phi_\lambda(\omega)  \) are determined up to \( O(1/\lambda)  \)
\end{claim}
\begin{proof}
Assume \( \omega  \) starts with \( H^N \). Due to our assumption that intermediate points lie in \( A  \), we know that
\[  y + N \lambda (1 - |x|) = [ML - 1, ML + 1]  \]
therefore
\[ 1 - \left| x \right| \in \left[ \frac{ML - 1 - y}{N \lambda}, \frac{ML + 1 - y}{N \lambda} \right] \]
or
\[ 1 - \left| x \right| = \frac{ML}{N\lambda} + O(\frac{1}{\lambda}) \]

Same calculation shows this for words starting with \( V  \).
\end{proof}

\pagebreak
\section{Actions}
\label{sec-3}
\begin{claim}
Given a fixed word \( \omega \in \mathbb{F}_2 \left\langle H,V \right\rangle  \) and a compatible homotopy class $\alpha$, all the actions relative to the loop having (alternatingly) \( x=0,y=0  \) are determined up to \( O(1) \).
\end{claim}
\begin{proof}
For each part of type \( H^N  \) or \( V^N  \) the action is
\[ \mathsf{A} = N F(x) - N \lambda x (1- \left| x \right|) \]
Now, \( F(x) = \lambda x (1 - \frac{\left| x \right|}{2})  \) and therefore
\[ \mathsf{A} = \lambda N x(1 - \frac{\left| x \right|}{2}) - \lambda N x + N \lambda \left| x \right| = N \lambda x \frac{\left| x \right|}{2} \]

For positive \( x  \) we get:

\[ \mathsf{A} = N \lambda \frac{x^2}{2} = \frac{N \lambda}{2} \left( 1 - \frac{ML}{N \lambda} - O(\frac{1}{\lambda}) \right)^2 = \frac{N \lambda}{2} - M L = - \frac{N \lambda}{2} + O(1)  \]

For a general word, we have a sum of such terms, up to double counting the area at the corners. However, there is a fixed amount of corners depending on the word, and the area of each corner is bounded by 4.
\end{proof}
% Emacs 24.4.1 (Org mode 8.2.10)
\end{document}
